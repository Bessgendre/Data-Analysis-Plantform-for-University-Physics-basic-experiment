\documentclass[UTF8]{ctexart}
\usepackage{graphicx}
\usepackage{amsmath}
\usepackage{cite}
\usepackage{mathtools}
\usepackage[a4paper,left=2.5cm,right=2.5cm,top=3cm,bottom=3cm]{geometry}
\title{密立根油滴实验:实验报告}
\author{禤科材 PB20030874 20级14系 707组 1号台}
\date{\today}
\bibliographystyle{plain}

\begin{document}
    \maketitle
    \tableofcontents


    \section{实验目的}
    \begin{center}
        \zihao{4}通过测定电场中油滴带电量测定单位元电荷
    \end{center}
    \section{实验原理}
    在密立根油滴实验中,测量电子电荷的基本设计思想是,使带电油滴在测量范围内处于受力平衡状态。按油滴作匀速或静止两种运动状态分类,可分为动态测量法和平衡测量法。本实验使用平衡测量法测元电荷。
    重力场中一个小的油滴,半径为 ,质量为 。空气是粘滞性流体,故此运动的油滴,除受重力和浮力外,还受粘滞阻力的作用。由斯托克斯定律,粘滞阻力与物体运动速度成正比。设油滴以均匀速度 下落,则有:
    \begin{equation}
        m_1g-m_2g=K_f
    \end{equation}

    此处$m_2$为与油滴同体积的空气的质量,$k$为比例系数,$g$为重力加速度。油滴在空气及重力场中
    的受力情况如下图所示:
    \begin{figure}[ht]
        \centering 
        \includegraphics[width=5cm]{z.pdf}
    \end{figure}

    \section{实验步骤}

    \section{实验记录}

    \section{数据处理}
        
    \section{误差分析}

    \section{提出改进}

    \nocite{a}
    \nocite{b}
    \nocite{c}
    \nocite{d}
    \bibliography{math}

\end{document}

